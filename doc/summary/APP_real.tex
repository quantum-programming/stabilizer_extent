\providecommand{\main}{.}
\documentclass[\main/main]{subfiles}
\setboolean{isMain}{false}

\begin{document}

\ifthenelse{\boolean{isMain}}{}{
    \begin{restatable}{theorem}{restrictedRealProblem}
        \label{thm:restrictedRealProblem}
        Suppose that the state $\ket{\psi}$ is real.
        If we substitute the matrix $A_n$
        with the matrix $A_n'$
        in the problem~\eqref{eq:stabilizerExtentPrimal},
        the optimal solution of the restricted problem
        is also optimal for the original problem.
    \end{restatable}
}

\section{The proof of Theorem~\ref{thm:restrictedRealProblem}}

In this section, we prove Theorem~\ref{thm:restrictedRealProblem}.
As we have stated in section~\ref{sec:restrictedRealProblem},
by substituting the matrix $A_n$ with the matrix $A_n'$,
we consider the restricted problems of
primal problem~\eqref{eq:stabilizerExtentPrimal}
and dual problem~\eqref{eq:stabilizerExtentDual}.
Let $x^*$ and $y^*$ be the optimal solutions of the restricted
primal and dual problems, respectively.
We can assure such solutions always exists.
Now, we show that the $x^*, y^*$ are
optimal not only for the restricted problems
but also for the original problems.

\begin{lemma}{\label{lem:absRealForTIsAbsRealForS}}
    Suppose $y$ is a real vector and satisfies
    $\abs{a^\dagger y} \leq 1$ for all $a \in \calA_n$
    such that $a$ is a real vector.
    Then, $y$ satisfies
    $\abs{a^\dagger y} \leq 1$ for all $a \in \calA_n$.
\end{lemma}
\begin{proof}
    We check the all states $\ket{\phi_i} \in \calS_n$ respectively.
    It is trivial for the case $k=0$ since the corresponding columns both exist in $\calS_n$ and $\calT_n$.
    We set $\ket{\phi_i} = \frac{1}{2^{k/2}}\sum_{x=0}^{2^k-1} (-1)^{x^\top Q x} i^{c^\top x} \ket{Rx+t}$ with $k>0$
    and $\braket{\phi_i}{y} = \alpha + i\beta (\alpha, \beta \in \bbR)$.
    The following two states
    \begin{equation*}
        \ket{\phi_+} \defeq \frac{1}{2^{k/2}}\sum_{x=0}^{2^k-1} (-1)^{x^\top Q x} \ket{Rx+t}, \quad
        \ket{\phi_-} \defeq \frac{1}{2^{k/2}}\sum_{x=0}^{2^k-1} (-1)^{x^\top Q x + c^\top x} \ket{Rx+t}
    \end{equation*}
    are in $\calT_n$, and satisfy $\braket{\phi_+}{y} = \alpha+\beta, \braket{\phi_-}{y} = \alpha-\beta$.
    From the assumption, we have
    \begin{equation*}
        \abs{\braket{\phi_i}{y}}
        =\sqrt{\alpha^2 + \beta^2}
        \leq \abs{\alpha}+\abs{\beta}
        = \max\{\abs{\alpha+\beta}, \abs{\alpha-\beta}\}
        \leq 1,
    \end{equation*}
    which completes the proof.
\end{proof}

\begin{theorem}
    The optimal solutions for the restricted problems $x^*$ and $y^*$ are also optimal for the original problems.
\end{theorem}
\begin{proof}
    Let $\mathrm{OPT}$ be the optimal value for the original primal problem.
    Since $x^*$ can be a feasible solution for the original primal problem, it is clear that $\mathrm{OPT} \leq \norm{x^*}_1$.
    By the strong duality theorem, $\mathrm{OPT}$ is also the optimal value for the original dual problem.
    From the lemma \ref{lem:absRealForTIsAbsRealForS},
    we can see that $y^*$ is a feasible solution for the original dual problem and $\mathrm{OPT} \geq \braket{\psi}{y^*}$.
    Again, by applying the strong duality theorem to the restricted problems,
    we have $\norm{x^*}_1 = \braket{\psi}{y^*}$, which means that $\mathrm{OPT} = \norm{x^*}_1 = \braket{\psi}{y^*}$.
    Therefore, $x^*$ and $y^*$ are also optimal for the original problems.
\end{proof}

\end{document}