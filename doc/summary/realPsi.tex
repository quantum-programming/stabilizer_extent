\providecommand{\main}{.}
\documentclass[\main/main]{subfiles}
\setboolean{isMain}{false}

\begin{document}

\section{For the Case \texorpdfstring{$\ket{\psi}$}{psi} is Real}

Assume that the state $\ket{\psi}$ is real.
Now, we consider the following optimization problem:
\begin{mini*}
    {x \in \mathbb{C}^\abs{\calS_n}}
    {\norm{x}_1}
    {\label{prob:primal}}
    {}
    \addConstraint{\sum_{\phi_i \in \calS_n}x_i\ket{\phi_i}}{= \ket{\psi}}{}
\end{mini*}
\begin{maxi*}
    {\ket{y} \in \mathbb{C}^{2^n}}
    {\Re(\braket{\psi}{y})}
    {\label{prob:dual}}
    {}
    \addConstraint{\abs{\braket{\phi_i}{y}}}{\leq 1,\quad}{\forall \phi_i \in \calS_n}
\end{maxi*}

Let $\calT_n$ be the collection of the states in $\calS_n$ with real coefficients.
This means $\calT_n$ is the collection of the states with $\bm{c}=0$ and
$\abs{\calT_n} = \abs{\calS_n}/(2^n)$.
Now, we consider the following restricted optimization problem:
\begin{mini*}
    {x \in \mathbb{R}^\abs{\calT_n}}
    {\norm{x}_1}
    {}{}
    \addConstraint{\sum_{\phi_j \in \calT_n}x_j\ket{\phi_j}}{= \ket{\psi}}{}
\end{mini*}
\begin{maxi*}
    {\ket{y} \in \mathbb{R}^{2^n}}
    {\braket{\psi}{y}}
    {}{}
    \addConstraint{\abs{\braket{\phi_j}{y}}}{\leq 1,\quad}{\forall \phi_j \in \calT_n}
\end{maxi*}

Let $x^*$ and $y^*$ be the optimal solutions of the restricted primal and dual problems, respectively.
These vectors always exist since $\calT_n$ forms a over complete basis.
We now show that the $x^*, y^*$ are
optimal not only for the restricted problems but also for the original problems.

\begin{lemma}{\label{lem:absRealForTIsAbsRealForS}}
    Suppose $\ket{y}$ is real and satisfies $\abs{\braket{\phi_j}{y}} \leq 1$ for all $\phi_j \in \calT_n$.
    Then, $\ket{y}$ satisfies $\abs{\braket{\phi_i}{y}} \leq 1$ for all $\phi_i \in \calS_n$.
\end{lemma}
\begin{proof}
    We check the all states $\ket{\phi_i} \in \calS_n$ respectively.
    It is trivial for the case $k=0$ since the corresponding columns both exist in $\calS_n$ and $\calT_n$.
    We set $\ket{\phi_i} = \frac{1}{2^{k/2}}\sum_{x=0}^{2^k-1} (-1)^{x^\top Q x} i^{c^\top x} \ket{Rx+t}$ with $k>0$
    and $\braket{\phi_i}{y} = \alpha + i\beta (\alpha, \beta \in \bbR)$.
    The following two states
    \begin{equation*}
        \ket{\phi_+} \defeq \frac{1}{2^{k/2}}\sum_{x=0}^{2^k-1} (-1)^{x^\top Q x} \ket{Rx+t}, \quad
        \ket{\phi_-} \defeq \frac{1}{2^{k/2}}\sum_{x=0}^{2^k-1} (-1)^{x^\top Q x + c^\top x} \ket{Rx+t}
    \end{equation*}
    are in $\calT_n$, and satisfy $\braket{\phi_+}{y} = \alpha+\beta, \braket{\phi_-}{y} = \alpha-\beta$.
    From the assumption, we have
    \begin{equation*}
        \abs{\braket{\phi_i}{y}}
        =\sqrt{\alpha^2 + \beta^2}
        \leq \abs{\alpha}+\abs{\beta}
        = \max\{\abs{\alpha+\beta}, \abs{\alpha-\beta}\}
        \leq 1,
    \end{equation*}
    which completes the proof.
\end{proof}

\begin{theorem}
    The optimal solutions for the restricted problems $x^*$ and $y^*$ are also optimal for the original problems.
\end{theorem}
\begin{proof}
    Let $\mathrm{OPT}$ be the optimal value for the original primal problem.
    Since $x^*$ can be a feasible solution for the original primal problem, it is clear that $\mathrm{OPT} \leq \norm{x^*}_1$.
    By the strong duality theorem, $\mathrm{OPT}$ is also the optimal value for the original dual problem.
    From the lemma \ref{lem:absRealForTIsAbsRealForS},
    we can see that $y^*$ is a feasible solution for the original dual problem and $\mathrm{OPT} \geq \braket{\psi}{y^*}$.
    Again, by applying the strong duality theorem to the restricted problems,
    we have $\norm{x^*}_1 = \braket{\psi}{y^*}$, which means that $\mathrm{OPT} = \norm{x^*}_1 = \braket{\psi}{y^*}$.
    Therefore, $x^*$ and $y^*$ are also optimal for the original problems.
\end{proof}

\end{document}