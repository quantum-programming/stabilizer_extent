\providecommand{\main}{.}
\documentclass[\main/main]{subfiles}
\setboolean{isMain}{false}

\begin{document}

\section{Derivation of the Dual Problem by CVXPY formulation}

In the \href{https://cvxopt.org/userguide/coneprog.html#cvxopt.solvers.socp}{cvxpy official userguide},
the following problem is considered.
We change the index from $k$ to $i$ for consistency with our notation.

\begin{mini*}
    {\bm{x}}
    {\bm{c}^\top \bm{x}}
    {}{}
    \addConstraint{G_i \bm{x} + \bm{s}_i}{= \bm{h}_i,\quad}{i = 0, \dots, M}
    \addConstraint{A\bm{x}}{= \bm{b}}{}
    \addConstraint{\bm{s}_0}{\succeq \bm{0}}{}
    \addConstraint{s_{i0}}{\geq \norm{\bm{s}_{i1}}_2,\quad}{i = 1, \dots, M}
\end{mini*}

\begin{maxi*}
    {\bm{y}}
    {-\sum_{i=0}^{M} \bm{h}_i^\top \bm{z}_i - \bm{b}^\top \bm{y}}
    {}{}
    \addConstraint{\sum_{i=0}^{M} G_i^\top \bm{z}_i + A^\top \bm{y} + \bm{c}}{= \bm{0}}{}
    \addConstraint{\bm{z}_0}{\succeq \bm{0}}{}
    \addConstraint{\bm{z}_{i0}}{\geq \norm{\bm{z}_{i1}}_2,\quad}{i = 1, \dots, M}
\end{maxi*}

where $\succeq$ denotes element-wise inequality, and
\begin{align*}
    \bm{s}_i & =(s_{i0}, \bm{s}_{i1}) \in \mathbb{R} \times \mathbb{R}^{r_i-1}, \\
    \bm{z}_i & =(z_{i0}, \bm{z}_{i1}) \in \mathbb{R} \times \mathbb{R}^{r_i-1}.
\end{align*}

We adapt this problem to our problem, minimizing the complex $L^1$ norm.
\begin{mini*}
    {\bm{x}\in \mathbb{C}^\abs{\calS_n}}
    {\norm{\bm{x}}_1}
    {}{}
    \addConstraint{A\bm{x}}{= \bm{b}}{}
\end{mini*}

Let $e_i$ be a vector with the $i$-th component 1 and the others 0.
Then, the problem can be written as follows.

\begin{mini*}
    {\bm{t}\in \mathbb{R}^\abs{\calS_n}, \bm{x}\in \mathbb{C}^\abs{\calS_n}}
    {\mqty[\bm{1}^\top & \bm{0}^\top & \bm{0}^\top] \mqty[\bm{t}\\ \Re(\bm{x})\\ \Im(\bm{x})]}
    {}{}
    \addConstraint{-\mqty[
            e_i^\top&\bm{0}^\top&\bm{0}^\top\\
            \bm{0}^\top&e_i^\top&\bm{0}^\top\\
            \bm{0}^\top&\bm{0}^\top&e_i^\top\\
        ] \mqty[\bm{t}\\ \Re(\bm{x})\\ \Im(\bm{x})] + \bm{s}_i}{= \mqty(0\\0\\0),\quad}{i = 1, \dots, \abs{\calS_n}}
    \addConstraint{\mqty[
            0_{2^n \times \abs{\calS_n}} & \Re(A) && -\Im(A)\\
            0_{2^n \times \abs{\calS_n}} & \Im(A) && \Re(A)
        ]\mqty[\bm{t}\\ \Re(\bm{x})\\ \Im(\bm{x})]}{= \mqty[\Re(\bm{b})\\ \Im(\bm{b})]}{}
    \addConstraint{s_{i0}}{\geq \norm{\bm{s}_{i1}}_2,\quad}{i = 1, \dots, \abs{\calS_n}}
\end{mini*}

\begin{maxi*}
    {\bm{y}\in \mathbb{C}^{2^n}}
    {-\mqty[\Re(\bm{b})^\top & \Im(\bm{b})^\top] \mqty[\Re(\bm{y})\\ \Im(\bm{y})] }
    {}{}
    \addConstraint{
    -\sum_{i=1}^{M}{\mqty[
        e_i & \bm{0} & \bm{0}\\
        \bm{0} & e_i & \bm{0}\\
        \bm{0} & \bm{0} & e_i
    ] \mqty[z_{i0}\\\Re(z_{i1})\\\Im(z_{i1})]}+ \mqty[
        0_{\abs{\calS_n} \times 2^n} & 0_{\abs{\calS_n} \times 2^n}\\
        \Re(A)^\top & \Im(A)^\top\\
        -\Im(A)^\top & \Re(A)^\top
    ]\mqty[\Re(\bm{y})\\ \Im(\bm{y})] +\mqty[\bm{1}\\ \bm{0}\\ \bm{0}]
    }{= \bm{0}}{}
    \addConstraint{z_{i0}}{\geq \abs{z_{i1}},\quad}{i = 1, \dots, \abs{\calS_n}}
\end{maxi*}

Furthermore, let the $i$-th column of the matrix $A$ be $\bm{a}_i$.
Then, the problem can be written as follows.
\begin{maxi*}
    {\bm{y}\in \mathbb{C}^{2^n}}
    {-\Re(\bm{b}^\dagger \bm{y})}
    {}{}
    \addConstraint{z_{i0}}{= 1,\quad}{i = 1, \dots, \abs{\calS_n}}
    \addConstraint{z_{i1}}{= \bm{a}_i^\dagger \bm{y},\quad}{i = 1, \dots, \abs{\calS_n}}
    \addConstraint{z_{i0}}{\geq \abs{z_{i1}},\quad}{i = 1, \dots, \abs{\calS_n}}
\end{maxi*}

\begin{maxi*}
    {\bm{y}\in \mathbb{C}^{2^n}}
    {-\Re(\bm{b}^\dagger \bm{y})}
    {}{}
    \addConstraint{\abs{\bm{a}_i^\dagger \bm{y}}}{\leq 1,\quad}{i = 1, \dots, \abs{\calS_n}}
\end{maxi*}

Even if we flip the sign of the objective function, the optimal solution does not change
thanks to the absolute value in the constraints.
Thus, the dual problem of the original problem is
\begin{maxi}
    {\bm{y}\in \mathbb{C}^{2^n}}
    {\Re(\bm{b}^\dagger \bm{y})}
    {\label{prob:CVXPYdual}}
    {}
    \addConstraint{\abs{\bm{a}_i^\dagger \bm{y}}}{\leq 1,\quad}{i = 1, \dots, \abs{\calS_n}}.
\end{maxi}

\section{Formulation by Mosek}

In the \href{https://docs.mosek.com/latest/pythonapi/tutorial-cqo-shared.html}{mosek official tutorial},
the following problem is considered with $\calD$ is a cone.
\begin{mini*}
    {\bm{x}}
    {\bm{c}^\top \bm{x}+\bm{c}^f}
    {}{}
    \addConstraint{l^c\leq A\bm{x}}{\leq u^c}{}
    \addConstraint{l^x\leq \phantom{A}\bm{x}}{\leq u^x}{}
    \addConstraint{F\bm{x}+g}{\in \calD}
\end{mini*}

For our problem, the problem can be written as follows.
\begin{mini*}
    {\bm{t}\in \mathbb{R}^\abs{\calS_n}, \bm{x}\in \mathbb{C}^\abs{\calS_n}}
    {\mqty[\bm{1}^\top & \bm{0}^\top & \bm{0}^\top] \mqty[\bm{t}\\ \Re(\bm{x})\\ \Im(\bm{x})] + 0}
    {}{}
    \addConstraint{
        \mqty[\Re(\bm{b})\\ \Im(\bm{b})]
        \leq
        \mqty[
            0_{2^n \times \abs{\calS_n}} & \Re(A) && -\Im(A)\\
            0_{2^n \times \abs{\calS_n}} & \Im(A) && \Re(A)
        ]\mqty[\bm{t}\\ \Re(\bm{x})\\ \Im(\bm{x})]}{
        \leq
        \mqty[\Re(\bm{b})\\ \Im(\bm{b})]
    }
    \addConstraint{-\infty \leq \mqty[\bm{t}\\ \Re(\bm{x})\\ \Im(\bm{x})]}{\leq \infty}
    \addConstraint{\mqty[I_{3\abs{\calS_n}}] \mqty[\bm{t}\\ \Re(\bm{x})\\ \Im(\bm{x})] + \bm{0}}{\in \qty(\calQ^3)^{\abs{\calS_n}} \quad (\text{with reordering})}
\end{mini*}

\section{Derivation of the Dual Problem by Lagrangian}

We can derive the dual problem more directly using the Lagrangian.
The original problem was formulated as follows.
\begin{mini}
    {\bm{x}\in \mathbb{C}^\abs{\calS_n}, \bm{t}\in \mathbb{R}^\abs{\calS_n}}
    {\sum_{i=1}^{\abs{\calS_n}} t_i}
    {\label{prob:primalForLagrangian}}
    {}
    \addConstraint{\mqty[
            \Re(A) & -\Im(A)\\
            \Im(A) & \Re(A)
        ]\mqty[
            \Re(\bm{x})\\
            \Im(\bm{x})
        ]}{= \mqty[
            \Re(\bm{b})\\
            \Im(\bm{b})
        ]}
    \addConstraint{\abs{x_i}}{\leq t_i,\quad}{i = 1, \dots, \abs{\calS_n}}
\end{mini}
The Lagrangian of the problem \eqref{prob:primalForLagrangian} can be defined by
\begin{align*}
    L(\bm{x}, \bm{t}, \bm{y},\bm{s})
    = & \sum_{i=1}^{\abs{\calS_n}} t_i                                                           \\
      & + \Re(\bm{y}^\top) (\Re(A) \Re(\bm{x}) - \Im(A) \Im(\bm{x}) - \Re(\bm{b}))               \\
      & + \Im(\bm{y}^\top) (\Im(A) \Re(\bm{x}) + \Re(A) \Im(\bm{x}) - \Im(\bm{b}))               \\
      & + \sum_{i=1}^{\abs{\calS_n}} s_i \qty(\abs{x_i} - t_i)                                   \\
    = & -\Re(\bm{y}^\dagger \bm{b})                                                              \\
      & + \sum_{i=1}^{\abs{\calS_n}} (1-s_i) t_i + \sum_{i=1}^{\abs{\calS_n}} \qty(s_i \abs{x_i}
    +\mqty[\Re(\bm{a}_i^\dagger \bm{y})                                                          \\
        \Im(\bm{a}_i^\dagger \bm{y})]^\top
    \mqty[\Re(\bm{x})                                                                            \\
        \Im(\bm{x})]).
\end{align*}
By using \href{https://math.stackexchange.com/questions/2738165/dual-of-a-second-order-cone-program-socp}{this relation},
we can obtain the Lagrange dual function as follows.
\begin{align*}
    g(\bm{y}, \bm{s})
    = & \inf_{\bm{x}, \bm{t}} L(\bm{x}, \bm{t}, \bm{y},\bm{s})                                       \\
    = & -\Re(\bm{b}^\dagger \bm{y})                                                                  \\
      & +\begin{cases}
             0       & \text{if } s_i \neq 1 \\
             -\infty & \text{otherwise}
         \end{cases}                                                             \\
      & +\begin{cases}
             0       & \text{if } \abs{\bm{a}_i^\dagger \bm{y}} \leq 1 \\
             -\infty & \text{otherwise}
         \end{cases}
\end{align*}
% 不等式制約に対するLagrange変数であることから要請される
% $s_i \geq 0$という制約は、$s_i = 1$より満たされることに注意する(厳密な言い回しか自信なし)。
Thus, the dual problem of the original problem is
\begin{maxi*}
    {\bm{y}\in \mathbb{C}^{2^n}}
    {-\Re(\bm{b}^\dagger \bm{y})}
    {}{}
    \addConstraint{\abs{\bm{a}_i^\dagger \bm{y}}}{\leq 1,\quad}{i = 1, \dots, \abs{\calS_n}},
\end{maxi*}
and even if we flip the sign of the objective function, the optimal solution does not change
thanks to the absolute value in the constraints.
Thus, the dual problem of the original problem is
\begin{maxi}
    {\bm{y}\in \mathbb{C}^{2^n}}
    {\Re(\bm{b}^\dagger \bm{y})}
    {\label{prob:dual}}
    {}
    \addConstraint{\abs{\bm{a}_i^\dagger \bm{y}}}{\leq 1,\quad}{i = 1, \dots, \abs{\calS_n}},
\end{maxi}
which is consistent with the dual problem \eqref{prob:CVXPYdual}.

\ifthenelse{\boolean{isMain}}{}{
    \bibliographystyle{quantum}
    \bibliography{stabilizerExtent}
}

\end{document}
