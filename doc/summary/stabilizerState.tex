\providecommand{\main}{.}
\documentclass[\main/main]{subfiles}
\setboolean{isMain}{false}

\begin{document}

\section{Enumeration of the Stabilizer States}

\begin{proposition}[{
                \cite[Theorem 2]{struchalinExperimentalEstimationQuantum2021b},
                \cite[Theorem 5.(ii)]{dehaeneCliffordGroupStabilizer2003},
                \cite{nestClassicalSimulationQuantum2010}
            }]
    All stabilizer states can be written as follows:
    \begin{equation}\label{eq:stabilizerStateStandardForm}
        \begin{dcases}
            \ket{\phi} \defeq \ket{t}                                                                       & \text{if $k=0$}, \\
            \ket{\phi} \defeq \frac{1}{2^{k/2}} \sum_{x=0}^{2^k-1}(-1)^{x^\top Q x} i^{c^\top x}\ket{R x+t} & \text{if $k>0$},
        \end{dcases}
    \end{equation}
\end{proposition}
\begin{proof}
    By hamada?
    In particular, can we say that all states in this form are stabilizer states?
\end{proof}

A little modification of the above proposition gives us
a efficient way to enumerate all the stabilizer states.

\begin{theorem}
    In order to enumerate all stabilizer states,
    it is enough to consider the cases satisfying the following conditions:
    \begin{quote}
        \begin{itemize}
            \item $Q$ is a top-left $\bbF_2^{k \times k}$ matrix.
            \item $R$ is a $\rank{k}$ $\bbF_2^{k \times (n-k)}$ rref(reduced row echelon form) matrix.
            \item $t$ belongs to the complement of the row space of $R$.
        \end{itemize}
    \end{quote}
\end{theorem}
\begin{proof}
    Main Ideas come from \cite{struchalinExperimentalEstimationQuantum2021b}.
    What we have to check is that
    this formulation can cover all the stabilizer states.
    It is easy to check that if $(Q_1, R_1, t_1) \neq (Q_2, R_2, t_2)$,
    then the corresponding states are also different,
    so we only have to check the number of stabilizer states.
    It is known that the number of $\rank k$ $\bbF_2^{k \times (n-k)}$ rref matrices is
    $\qBinom{n}{k}_2$, which is a q-binomial coefficient with $q=2$.
    Thus, The number of $Q,c,R,t$ is
    $2^{k(k+1)/2},2^k,\qBinom{n}{k}_2,2^{n-k}$, respectively,
    and the total number of states is
    \begin{equation*}
        2^n + \sum_{k=1}^{n} 2^{k(k+1)/2} 2^k \qBinom{n}{k}_2 2^{n-k} \\
        = 2^n \sum_{k=0}^{n} \qBinom{n}{k}_2 2^{k(k+1)/2}               \\
        = 2^n \prod_{k=1}^{n} (2^k+1)
        = \abs{\calS_n}.
    \end{equation*}
    In the second last equation, we used the q-binomial theorem.
    Therefore, this formulation actually covers all the stabilizer states.
\end{proof}

In the above theorem, we used $\bbF_2$.
By doing so, we can separate the coefficients of $-1$ and $i$
since $i^0=1,i^1=i$, without no appearance of $-1$.
This is a nice property, but at the same time,
the law of exponents does not hold due to $\bbF_2$,
i.e., $1+1=0$ in $\bbF_2$ but $-1 = i^{1+1} \neq i^{0} = 1$.
This fact encourages us to allow $c^\top x$ to take non negative integer values,
and here is another formulation with a slightly difference
in order to solve this problem.
\begin{corollary}\label{cor:stabilizerStateStandardFormWithZ}
    In the above theorem,
    We can change $\bbF_2$ to $\{0,1\} \subset \bbZ$.
\end{corollary}
\begin{proof}
    We only have to check the term $i^{c^\top x}$,
    since other terms are the same as the above theorem.
    By changing $\bbF_2$ to $\{0,1\} \subset \bbZ$,
    the term $i^{c^\top x}$ change iff $p \equiv 2,3 \pmod 4$,
    where $p$ is the number of $i$ such that $c_i=1$ and $x_i=1$.
    By flipping the value of $Q_{ij}$ iff $c_i=c_j=1(i \neq j)$,
    we can flip this negative term, since
    \begin{align*}
        \binom{p}{2} \equiv \begin{cases}
                                0 \pmod 2 & \text{if $p \equiv 0,1 \pmod 4$}, \\
                                1 \pmod 2 & \text{if $p \equiv 2,3 \pmod 4$}.
                            \end{cases}
    \end{align*}
\end{proof}

\section{Calculating the Overlap}

Thanks to the corollary \ref{cor:stabilizerStateStandardFormWithZ},
we can prove the following theorem.
\begin{theorem}
    Fix $k,R,t$ in the standard form \eqref{eq:stabilizerStateStandardForm}.
    Then, we can compute the overlap $\braket{\phi}{\psi}$ efficiently.
    (TODO: Write the exact computational cost.)
\end{theorem}
\begin{proof}
    (Following is rough and crude proof.)

    We only consider the case $k>0,R=0,t=0$ for the simplicity.
    Other cases are trivial or can be reduced to this case.
    Define $x \defeq \begin{bmatrix}
            x_0 \\
            \overline{x}
        \end{bmatrix}$,
    $c \defeq \begin{bmatrix}
            c_0 \\
            \overline{c}
        \end{bmatrix}$, and
    $Q \defeq \begin{bmatrix}
            Q_{00} & Q_{0}^\top   \\
            0      & \overline{Q}
        \end{bmatrix}$
    ($x_0,c_0$ and $Q_{00}$ are all in $\{0,1\}$).
    Since
    $x^\top Q x = x_0 (Q_{00}+Q_0^\top \overline{x}) + \overline{x}^\top \overline{Q} \overline{x}$
    and
    $c^\top x = c_0 x_0 + \overline{c}^\top \overline{x}$,
    we can rewrite the state as
    \begin{align*}
        \ket{\phi} & = \sum_{x=0}^{2^k-1} (-1)^{x^\top Q x} i^{c^\top x} \ket{x}                                                                                   \\
                   & = \sum_{\overline{x}=0}^{2^{k-1}-1} (-1)^{\overline{x}^\top \overline{Q} \overline{x}} i^{\overline{c}^\top \overline{x}}
        \qty(\ket{2\overline{x}} + (-1)^{Q_{00}+Q_0^\top \overline{x}} i^{c_0} \ket{2\overline{x}+1})                                                              \\
                   & = \sum_{\overline{x}=0}^{2^{k-1}-1} (-1)^{\overline{x}^\top \overline{Q} \overline{x}} i^{\overline{c}^\top \overline{x}} \ket{\overline{x}'}
    \end{align*}
    by defining $\ket{\overline{x}'} \defeq \ket{2\overline{x}} + (-1)^{Q_{00}+Q_0^\top \overline{x}} i^{c_0} \ket{2\overline{x}+1}$.
    (Question: Is it natural to equate integer $2\overline{x}+1$ to the vector $\begin{bmatrix} 1 \\ \overline{x} \end{bmatrix}$?)

    Thus, we can compute the overlap recursively
    with very small computational cost per each step.
    This leads to the efficient calculation of the overlaps,
    which concludes the proof.
\end{proof}

\begin{proposition}
    % 枝刈りが効く
    For the each steps, we can skip the calculation of the overlap if the following conditions are satisfied:
    \begin{equation*}
        \sum_{x=0}^{2^k-1} \braket{Rx+t}{\psi} < \mathrm{threshold}
    \end{equation*}
\end{proposition}
\begin{proof}
    The overlap can be suppressed by $L^1$ norm of the state.
    (TODO: Write exact proof.)
\end{proof}

\ifthenelse{\boolean{isMain}}{}{
    \bibliographystyle{quantum}
    \bibliography{stabilizerExtent}
}

\end{document}
